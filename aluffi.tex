\documentclass[a4paper, 11pt]{book}
\usepackage{comment} % enables the use of multi-line comments (\ifx \fi) 
\usepackage{lipsum} %This package just generates Lorem Ipsum filler text. 
\usepackage{fullpage} % changes the margin
\usepackage[a4paper, total={7in, 10in}]{geometry}

%\usepackage{tgadventor} % The font for the entire document can be changed here
%\usepackage{courier}
%\usepackage{charter}
\usepackage{tgcursor}

\usepackage{mathrsfs} 
\usepackage{quiver} 
\newtheorem{corollary}{Corollary}
\usepackage{graphicx}
\usepackage{tikz}

\usetikzlibrary{arrows}
\usepackage{verbatim}
\usepackage[numbered]{mcode}
\usepackage{float}
\usepackage{tikz}
    \usetikzlibrary{shapes,arrows}
    \usetikzlibrary{arrows,calc,positioning}

    \tikzset{
        block/.style = {draw, rectangle,
            minimum height=1cm,
            minimum width=1.5cm},
        input/.style = {coordinate,node distance=1cm},
        output/.style = {coordinate,node distance=4cm},
        arrow/.style={draw, -latex,node distance=2cm},
        pinstyle/.style = {pin edge={latex-, black,node distance=2cm}},
        sum/.style = {draw, circle, node distance=1cm},
    }
\usepackage{xcolor}
\usepackage{mdframed}
\usepackage[shortlabels]{enumitem}
\usepackage{indentfirst}
\usepackage{hyperref}
\usepackage{amsmath,amsfonts,amsthm, amssymb}
\usepackage{array}
\usepackage[all,textures]{xy}
\usepackage{graphicx}
\usepackage{alltt}
\usepackage{listings}
\usepackage{float}
\usepackage{tabu}
\usepackage{longtable}

\theoremstyle{plain}
\newtheorem{exercise}{Exercise}
\newtheorem*{theorem1}{Theorem 1}
\newtheorem*{notation}{Notation}
\newtheorem*{corollary1.1}{Corollary 1.1}
\newtheorem*{theorem2}{Theorem 2}
\newtheorem*{corollary2.1}{Corollary 2.1}
\newtheorem*{remarks}{Remarks}
\newtheorem*{theorem}{Theorem}
\newtheorem*{Corollary}{Corollary}

    
\renewcommand{\thesubsection}{\thesection.\alph{subsection}}


% Define solution environment

%%%%%%%%%%%%%%%%%%%%%%%%%%%%%%%%%%%%%%%%%%%%%%%%%%%%%%%%%%%%%%%%%%%%%%%%%%%%%%%%%%%%%%%%%%%%%%%%%%%%%%%%%%%%%%%%%%%%%%%%%%%%%%%%%%%%%%%% Original packages, custom environments, and custom commands below
  \usepackage{amsmath,amsfonts,amsthm, amssymb}
  \usepackage{fullpage}
  \usepackage{array}
  \usepackage[all,textures]{xy}
  \usepackage{graphicx}
  \usepackage{alltt}
  \usepackage{listings}
  \usepackage{float}
  \usepackage{tabu}
  \usepackage{longtable}
  \usepackage{lipsum}
  \usepackage[T1]{fontenc}
  
  \theoremstyle{plain}
  \newtheorem{innercustomgeneric}{\customgenericname}
\providecommand{\customgenericname}{}
\newcommand{\newcustomtheorem}[2]{%
  \newenvironment{#1}[1]
  {%
   \renewcommand\customgenericname{#2}%
   \renewcommand\theinnercustomgeneric{##1}%
   \innercustomgeneric
  }
  {\endinnercustomgeneric}
}

\newcustomtheorem{definition}{Definition}
\newcustomtheorem{lemma}{Lemma}

\newcommand{\mb}{\mathbf}
\newcommand{\arr}{\rightarrow}
\newcommand{\op}{\text{op}}
\newcommand{\inv}{{-1}}
\newcommand{\Z}{\mathbb{Z}}
\newcommand{\U}{\mathcal{U}}
\newcommand{\dom}{\text{dom}}
\newcommand{\p}{\prime}
\newcommand{\obj}{\text{obj}}
\newcommand{\id}{\text{Id}}
\newcommand{\mc}{\mathcal}
\newcommand{\warr}{\xrightarrow}
\newcommand{\tb}{\textbf}
\newcommand{\C}{\mathbf{C}}
\newcommand{\cod}{\text{cod}}
\newcommand{\Co}{\mathbf{C}^op}
\newcommand{\x}{\times}
\newcommand{\ran}{\text{ran}}
\newcommand{\D}{\mb{D}}
\newcommand{\F}{\mc{F}}
\newcommand{\G}{\mc{G}}
\newcommand{\iso}{\cong}
\newcommand{\tarr}{\twoheadrightarrow}
\newcommand{\ovl}{\overline}
\newcommand{\mon}{\rightarrowtail}
\newcommand{\la}{\langle} 
\newcommand{\ra}{\rangle}
\newcommand{\Hom}{\text{Hom}}
\newcommand{\N}{\mathbb{N}}
\newcommand{\fst}{\text{fst}} 
\newcommand{\snd}{\text{snd}}
\newcommand{\Arr}{\text{Arr}}
  \setlength{\parindent}{0pt}

  \setlength{\parindent}{0pt}
 
 
 \newtheorem{thm}{Exercise}
\begin{document}
  \begin{titlepage}
	\centering % Center everything on the title page
	\scshape % Use small caps for all text on the title page
	\vspace*{1.5\baselineskip} % White space at the top of the page
% ===================
%	Title Section 	
% ===================

	\rule{13cm}{1.6pt}\vspace*{-\baselineskip}\vspace*{2pt} % Thick horizontal rule
	\rule{13cm}{0.4pt} % Thin horizontal rule
	
		\vspace{0.75\baselineskip} % Whitespace above the title
% ========== Title ===============	
	{	\Huge Algebra: Chapter 0\\ 
			\vspace{4mm}
		Book by Paolo Aluffi \\	}
% ======================================
		\vspace{0.75\baselineskip} % Whitespace below the title
	\rule{13cm}{0.4pt}\vspace*{-\baselineskip}\vspace{3.2pt} % Thin horizontal rule
	\rule{13cm}{1.6pt} % Thick horizontal rule
	
		\vspace{1.75\baselineskip} % Whitespace after the title block
% =================
%	Information	
% =================
	{\large : 
		\vspace*{1.2\baselineskip}
	} \\
	\vfill

\end{titlepage}
%%%%%%%%%%%%%%%%%%%%%%%%%%%%%%%%%%%%%%%%%%%%%%%%%%%%%%%%%%%
\begingroup
\let\cleardoublepage\clearpage
\tableofcontents
\endgroup
\addcontentsline{toc}{chapter}{III. Rings and modules}
\chapter*{III. Rings and modules}
  \begin{exercise} (Section 1, 17)
    Explain in what sense $R[x]$ agrees with monoid ring $R[\N]$.
  \end{exercise}
  \begin{proof}
    We define homomorphism $f:R[x] \arr R[\N]$ so that $f \left( \sum_{i=0}^\infty a_i x^i \right) = \sum_{i=0}^\infty a_i \cdot i$. It is easy to see that is indeed a homomorphism. Define $g: R[\N] \arr R[x]$ as $g\left( \sum_{i=0}^\infty a_i i \right)=\sum_{i=0^\infty} a_i x^i$. Again, it is easy to see it is a homomorphism. We now prove that they are infact isomorphism. For any $\sum_{i=0}^\infty a_i x^i \in R[x]$, we have 
    \begin{align*}
      g \circ f \left(\sum_{i=0}^\infty a_i x^i \right) &= g \left(f \left(\sum_{i=0}^\infty a_i x^i \right) \right) \\
      &=g \left( \sum_{i=0}^\infty a_i \cdot i \right) \\
      &=\sum_{i=0}^\infty a_i x^i.
    \end{align*}
    Thus, $g \circ f=1_{R[x]}$. Similarly, we can show $f \circ g = 1_{R[\N]}$. Hence, $R[x] \simeq R[\N]$.
  \end{proof}

  \begin{exercise}
    (Section 2, 13) 
    Verify that the 'componentwise' product $R_1 \times R_2$ of two rings satisfies the universal property for products in a category.
  \end{exercise}
  \begin{proof}
    In general, the following result holds.
    \begin{theorem}
      Let $I$ be an indexed set. Consider some family of rings $(R_i)_{i \in I}$. Then the componentwise product $\prod_{i \in I} R_i$ satisfy the ump of indexed product in $\mb{Rings}$. 
    \end{theorem}
      Let's first define the ump of indexed product. \\
      \textbf{Unique mapping property of product of an indexed family.} Let $\C$ be a category with binary products. The object $\prod_{i \in I} X_i$ is said to be product of indexed family $(X_i)_{i \in I}$ if there exists morphism $\pi_i:\prod_{i \in I} X_i \arr X_i$ for each $i \in I$ and for any object $A$ such that there are morphism $x_i: A \arr X_i$ for each $i$, there exists unique arrow $f:A \arr \prod_{i \in I} X_i$ with the property that $x_i=\pi_i f$ for any $i \in I$. In other, the diagram 
      \[\begin{tikzcd}
        && A \\
        \\
        && {\prod_{i \in I} X_i} \\
        \\
        {X_j} &&&& {X_k}
        \arrow["{\exists !f}"{description}, dashed, from=1-3, to=3-3]
        \arrow["{\pi_j}"{description}, from=3-3, to=5-1]
        \arrow["{\pi_k}"{description}, from=3-3, to=5-5]
        \arrow["{x_j}"'{description}, from=1-3, to=5-1]
        \arrow["{x_k}"'{description}, from=1-3, to=5-5]
      \end{tikzcd}\]
      must commute for all $j,k \in I$.
\newpage
      Let $\pi_i: \prod_{i \in I} R_i \arr R_i$ be the projection homo. Consider a ring $S$ such that there are ring homo $s_i: S \arr R_i$ for each $i$. Define $f:S \arr \prod_{i \in I} R_i$ so that $x \mapsto (s_i(x))_{i \in I}$ for any $x \in S$. It is easy to see $f$ is unique homo making 
      \[\begin{tikzcd}
        && S \\
        \\
        && {\prod_{i \in I} R_i} \\
        \\
        {R_j} &&&& {R_k}
        \arrow["{\exists !f}"{description}, dashed, from=1-3, to=3-3]
        \arrow["{\pi_j}"{description}, from=3-3, to=5-1]
        \arrow["{\pi_k}"{description}, from=3-3, to=5-5]
        \arrow["{s_j}"'{description}, from=1-3, to=5-1]
        \arrow["{s_k}"'{description}, from=1-3, to=5-5]
      \end{tikzcd}\]
      commute for any $j,k \in I$. Hence, $\prod_{i \in I} R_i$ statisfy ump of product of indexed family. As an corollary of this result, we have $R_1 \times R_2$ satisfy ump of binary product.
    \end{proof}
  

  \begin{exercise}
    (Section 2, 14)
    Verify that $\Z[x_1,x_2]$ satisfies the universal property for the coproducts of two copies of $\Z[x]$ in the category of commutative rings. Explain why it does not satisfy it in $\mb{Rings}$.
  \end{exercise}
  \begin{proof}
    In general, we have the following result.
    \begin{theorem}
      Let $I$ be any index set and let $(\Z[x_i])_{i \in I}$ be indexed family of set of polynomial with integer coefficients. Then $\Z[x_i]_{i \in I}$ satisfy the appropriate unique mapping property of indexed family of coproduct. That is 
      $$ \coprod_{i \in I} \Z[x_i] \simeq \Z[x_i]_{i \in I}.$$
    \end{theorem}
    Before we get into the proof of it, let's first define the required unique mapping property! \\
\newpage
    \textbf{Unique mapping property of coproduct of an idexed family} Let $\C$ be a category with binary coproducts. The object $\coprod_{i \in I} X_i$ is said to be the product of indexed family $(X_i)_{i \in I}$ if there are inclusion morphism $e_i:X_i \arr \coprod_{i \in I} X_i$ for each $i \in I$, and for any object $A$ such that there are morphisms $x_i:X_i \arr A$ for each $i$, then there exists an unique morphism $f:\coprod_{i \in I} X_i \arr A$ with the property $f \circ e_i = x_i$. In other words, $f$ make 
    \[\begin{tikzcd}
      &&& A \\
      \\
      \\
      &&& {\coprod_{i \in I} X_i} \\
      \\
      {X_i} &&&&&& {X_j}
      \arrow["{\exists ! f}"{description}, dashed, from=4-4, to=1-4]
      \arrow["{x_i}"{description}, from=6-1, to=1-4]
      \arrow["{x_j}"{description}, from=6-7, to=1-4]
      \arrow["{e_i}"{description}, from=6-1, to=4-4]
      \arrow["{e_j}"{description}, from=6-7, to=4-4]
    \end{tikzcd}\]
    commute for any $i,j \in I$.

  Now, we will proceed to show that $\Z[x_i]_{i \in I}$ posses this UMP. It is easy to see that for each $i$, we have inclusion $e_i:\Z[x_i] \arr \Z[x_i]_{i \in I}$. Consider a ring $S$ such that there are morphism $s_i:\Z[x_i] \arr S$ for every $i$. Define $f:\Z[x_i]_{i \in I} \arr S$ so that 
  \begin{align*}
    f \left( \sum m_{j_i} x_i^{j_i} \right)_{i \in I} &= \sum f(m_{j_i} f (x_i)^{j_i}) \\
    &= \sum k(m_{j_i}) s_i(x_i^{j_i}) \\
  \end{align*}
  for any polynomial in $\Z[x_i]_{i \in I}$, where $k:\Z \arr S$ is the unique ring homomorphism (as $\Z$ is initial in $\mb{Rings}$). Note that the summation is necessarily finite. It is easy to see that $f$ is ring morphism and also unique one at that to satisfy 
  \[\begin{tikzcd}
    &&& S \\
    \\
    \\
    &&& {Z[x_i]_{i \in I}} \\
    \\
    {\Z[x_i]} &&&&&& {\Z[x_j]}
    \arrow["{\exists ! f}"{description}, dashed, from=4-4, to=1-4]
    \arrow["{s_i}"{description}, from=6-1, to=1-4]
    \arrow["{s_j}"{description}, from=6-7, to=1-4]
    \arrow["{e_i}"{description}, from=6-1, to=4-4]
    \arrow["{e_j}"{description}, from=6-7, to=4-4]
  \end{tikzcd}\]
  Hence, $\Z[x_i]_{i \in I}$ satisfy UMP of indexed family of coporducts of $(\Z[x_i])_{i \in I}$. As an easy corollary of the theorem, we have $\Z[x_1,x_2] \simeq \Z[x_1] \coprod \Z[x_2]$.
\end{proof}
\end{document}
